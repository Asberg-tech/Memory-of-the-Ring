\documentclass[a4paper,12pt]{article}
\usepackage[utf8]{inputenc}
\usepackage{graphicx}
\usepackage{hyperref}
\usepackage{geometry}
\usepackage{xcolor}
\usepackage{listings}
\usepackage{fancyvrb}
\usepackage{framed}

\geometry{a4paper, margin=1in}

% Definiera färger för kodblock
\definecolor{codebackground}{RGB}{40, 44, 52}
\definecolor{codetext}{RGB}{171, 178, 191}

% Skapa en anpassad miljö för kod
\DefineVerbatimEnvironment{codebox}{Verbatim}{
    frame=single,
    framesep=2mm,
    baselinestretch=1.2,
    fontsize=\small,
    commandchars=\\\{\},
    formatcom=\color{codetext},
    bgcolor=codebackground,
    breaklines=true,
    breakanywhere=true,
    breaksymbol=⤶,
    numbers=left,
    numbersep=5pt,
    xleftmargin=15pt
}

\title{\textbf{Gesällprov: Memory of the Ring}}
\author{Niklas Åsberg}
\date{\today}

\begin{document}

\maketitle

\section*{Sammanfattning}
Memory of the Ring är ett minnesspel inspirerat av fantasyvärlden där spelaren matchar kortpar med kända karaktärer och symboler. Spelet är utvecklat i HTML, CSS och JavaScript och har funktioner såsom kortvändning, matchningslogik, drag- och tidräknare samt en vinstskärm. Layouten är responsiv och temat är anpassat till den mytiska atmosfären.

\begin{figure}[h]
    \centering
    \includegraphics[width=0.8\textwidth]{images/start.png}
    \caption{Spelets startskärm}
    \label{fig:game_start}
\end{figure}

\section*{Framtagande}
Utvecklingsprocessen bestod av flera steg som innefattade planering, design och implementering.

\subsection*{Idé och Koncept}
Spelet bygger på ett minnesspel där spelaren matchar bilder av ikoniska karaktärer och föremål från en fantasyvärld. Målet är att göra spelet estetiskt tilltalande och intuitivt att spela.

\subsection*{Struktur och Planering}
Projektet strukturerades för att hålla koden organiserad och underhållbar:
\begin{itemize}
    \item \textbf{pages/} - HTML-filer
    \item \textbf{styles/} - CSS-filer
    \item \textbf{scripts/} - JavaScript-kod
    \item \textbf{images/} - Spelgrafik
\end{itemize}
Denna uppdelning säkerställde en tydlig separation av ansvar mellan olika koddelar.

\subsection*{Utvecklingssteg}
\begin{enumerate}
    \item \textbf{Initial Utveckling}:
    \begin{itemize}
        \item Implementerade grundläggande spelmekanik med korthantering
        \item Etablerade timer och move-counter funktionalitet
        \item Skapade grundläggande HTML-struktur med spelplan, titel och kontroller
        \item Valde att dynamiskt generera spelkorten med JavaScript för ökad flexibilitet
    \end{itemize}

    \item \textbf{Visuella Förbättringar}:
    \begin{itemize}
        \item Började med enkel lila bakgrund på korten
        \item Utvecklade till tematisk kortdesign med fantasy-inspirerade bilder
        \item Implementerade animationer för kortflipp och hover-effekter
        \item Skapade responsiv design med CSS Grid och Flexbox
        \item Utvecklade ett konsekvent färgtema med \texttt{colors.css}
    \end{itemize}

    \item \textbf{Kodstruktur och Organisation}:
    \begin{itemize}
        \item Startade med all kod i en fil för enkelhet
        \item Försökte separera Card-klassen till egen modul
        \item Återgick till single-file lösning på grund av CORS-begränsningar
        \item Implementerade objektorienterad design med Card och MemoryGame klasser
        \item Dokumenterade beslutsprocessen för framtida referens
    \end{itemize}

    \item \textbf{Bildhantering}:
    \begin{itemize}
        \item Genererade unika bilder med DALL·E 3 för att undvika upphovsrättsproblem
        \item Optimerade bildstorlekar och laddningstider
        \item Skapade konsekvent visuell stil över alla spelkort
        \item Implementerade responsiv bildhantering
    \end{itemize}

    \item \textbf{Förbättring av Vinstskärm}:
    \begin{itemize}
        \item Ersatte basic JavaScript alert med en styled modal
        \item Lade till gratulationsbild och tematisk typografi
        \item Implementerade spelstatistik (drag och tid)
        \item Skapade "Play Again"-funktionalitet
        \item Designade responsiv layout med animationer
    \end{itemize}

    \item \textbf{Användarupplevelseförbättringar}:
    \begin{itemize}
        \item Implementerade mjuka övergångar och animationer
        \item Lade till hover-effekter för bättre feedback
        \item Förbättrade kortflipp-mekaniken
        \item Optimerade prestanda för animationer
    \end{itemize}

    \item \textbf{Prestandaoptimering}:
    \begin{itemize}
        \item Implementerade lazy loading för bilder
        \item Optimerade CSS-animationer
        \item Förbättrade minneshantering
        \item Säkerställde jämn prestanda på olika enheter
    \end{itemize}

    \item \textbf{Dokumentation och Kvalitetssäkring}:
    \begin{itemize}
        \item Implementerade omfattande JSDoc dokumentation
        \item Skapade tydlig kodstruktur med väldefinierade klasser
        \item Säkerställde konsekvent kodstil
        \item Testade och åtgärdade buggar, särskilt kring kortvändningslogiken
    \end{itemize}
\end{enumerate}

\begin{figure}[h]
    \centering
    \includegraphics[width=0.8\textwidth]{images/play.png}
    \caption{Spelet under kortvändning}
    \label{fig:game_play}
\end{figure}

\section*{Funktion}
Spelet är enkelt att starta och spela:

\subsection*{Uppstart}
För att starta spelet behöver användaren bara öppna \texttt{index.html} i en webbläsare. Ingen installation eller webbserver krävs, vilket gör spelet mycket tillgängligt. Vid start presenteras användaren med spelplanen som visar:
\begin{itemize}
    \item En titel "Memory of the Ring" i medeltida typsnitt
    \item 16 nedvända kort arrangerade i ett 4x4 rutnät
    \item En räknare för antal drag
    \item En timer som startar vid första kortväljningen
    \item En "New Game" knapp för att starta om spelet
\end{itemize}

\begin{figure}[h]
    \centering
    \includegraphics[width=0.8\textwidth]{images/start.png}
    \caption{Startskärmen visar spelplanen med alla kort nedvända}
    \label{fig:game_start}
\end{figure}

\subsection*{Spelmekanik}
Spelet följer klassisk memory-mekanik med några moderna tillägg:
\begin{enumerate}
    \item Spelaren börjar genom att klicka på valfritt kort, vilket vänds upp och visar en fantasy-karaktär eller symbol.
    \item Ett andra kort väljs och vänds upp. Systemet jämför nu de två korten:
    \begin{itemize}
        \item Om korten matchar förblir de uppvända
        \item Om korten inte matchar vänds de automatiskt tillbaka efter en kort stund
    \end{itemize}
    \item För varje par av kort som vänds (oavsett om de matchar eller inte):
    \begin{itemize}
        \item Drag-räknaren ökar med ett
        \item Tiden fortsätter att ticka
    \end{itemize}
    \item Spelet fortsätter tills alla par har hittats
\end{enumerate}

\begin{figure}[h]
    \centering
    \includegraphics[width=0.8\textwidth]{images/play.png}
    \caption{Spelet under pågående match, med några kort uppvända och matchade}
    \label{fig:game_play}
\end{figure}

\subsection*{Vinsttillstånd}
När spelaren har hittat alla par:
\begin{itemize}
    \item Timern stannar automatiskt
    \item En animerad vinstskärm visas med:
    \begin{itemize}
        \item En gratulationsbild
        \item Spelarens totala antal drag (t.ex. "Du vann på 24 drag")
        \item Den totala speltiden (t.ex. "Tid: 01:45")
        \item En "Play Again" knapp för att starta ett nytt spel
    \end{itemize}
    \item Spelaren kan välja att starta ett nytt spel, där korten blandas om i nya positioner
\end{itemize}

\begin{figure}[h]
    \centering
    \includegraphics[width=0.6\textwidth]{images/win.png}
    \caption{Vinstskärmen visar spelarens resultat och möjlighet att spela igen}
    \label{fig:game_win}
\end{figure}

Denna speldesign uppmuntrar till återspelning genom att låta spelaren försöka förbättra sitt resultat genom att minska antalet drag eller tiden det tar att hitta alla par.

\section*{Form}
Koden är strukturerad enligt objektorienterade principer med två huvudklasser: \texttt{Card} och \texttt{MemoryGame}. Denna uppdelning möjliggör en tydlig separation av ansvar där varje klass hanterar sin specifika funktionalitet.

\subsection*{Klassstruktur}
\subsubsection*{Card-klassen}
Hanterar enskilda korts egenskaper och beteende:

\begin{codebox}
class Card:
    properties:
        id          // Unikt ID för matchning av par
        imagePath   // Sökväg till kortets bild
        name        // Kortets namn/beskrivning
        isFlipped   // Om kortet är vänt
        isMatched   // Om kortet har matchats

    methods:
        constructor(id, imagePath, name)
        createElement(index, flipCallback)
\end{codebox}

\subsubsection*{MemoryGame-klassen}
Hanterar spellogiken och spelflödet:

\begin{codebox}
class MemoryGame:
    properties:
        cards[]           // Array med alla kort
        flippedCards[]    // Array med vända kort
        matchedPairs      // Antal hittade par
        moves            // Antal drag
        gameStarted      // Om spelet har börjat
        timer            // Referens till timer
        seconds          // Spelad tid

    methods:
        constructor()
        initializeGame()
        shuffleCards()
        renderCards()
        flipCard(index)
        checkMatch()
        startTimer()
        showVictoryScreen()
        endGame()
        resetGame()
\end{codebox}

\subsection*{Centrala Kodfunktioner}
\subsubsection*{Korthantering}
Kortens vändningslogik implementeras genom CSS-transformationer och JavaScript-events:

\begin{codebox}
program flipCard
flipCard_args(1) = kortIndex

if kort_är_vänt(kortIndex) or antal_vända_kort() >= 2 then
    return "nekad vändning"

if första_draget() then
    startTimer()

vändKort(kortIndex)
läggTillVäntKort(kortIndex)

if antal_vända_kort() == 2 then
    ökaAntalDrag()
    kontrollera_match()

return "kort vänt"
\end{codebox}

\subsubsection*{Matchningskontroll}
Kontrollerar om två vända kort matchar:

\begin{codebox}
program checkMatch

kort1 = hämta_vänt_kort(1)
kort2 = hämta_vänt_kort(2)

if kort1.id == kort2.id then
    markera_matchade(kort1, kort2)
    öka_antal_par()
    if alla_par_hittade() then
        visa_vinstskärm()
else
    vänta(1)
    vänd_tillbaka(kort1, kort2)

töm_vända_kort()
return "matchkontroll klar"
\end{codebox}

\subsection*{Spelflöde}
Det övergripande spelflödet:

\begin{codebox}
program startGame

// Initiering
spelinstans = skapa_ny_spelinstans()
kortarray = generera_kort()
blanda_kort(kortarray)
rendera_spelplan(kortarray)

// Huvudloop
loop
    if användar_klick() then
        kortIndex = hämta_klickat_kort()
        flipCard(kortIndex)
        
    if alla_par_hittade() then
        exit

    uppdatera_ui()

// Spelavslut
stoppa_timer()
resultat = beräkna_slutresultat()
visa_vinstskärm(resultat)

return "spel avslutat"
\end{codebox}

\subsection*{Timer-hantering}
Timer för speltidsräkning:

\begin{codebox}
program startTimer

sekunder = 0
timer_aktiv = true

loop
    if !timer_aktiv then
        exit
        
    vänta(1)
    sekunder = sekunder + 1
    visa_tid(formatera_tid(sekunder))

return "timer stoppad"

program formatera_tid
formatera_tid_args(1) = sekunder

minuter = sekunder / 60
återstående_sekunder = sekunder % 60

return minuter + ":" + återstående_sekunder
\end{codebox}

\subsection*{Kortgenerering}
Skapar och blandar spelkorten:

\begin{codebox}
program generera_kort

kortlista = []
karaktärer = ["frodo", "gandalf", "aragorn", "legolas", 
              "sauron", "gollum", "ring", "gimli"]

for i = 1 to längd(karaktärer) step 1
    kort1() = skapa_kort(i, karaktärer(i))
    kort2() = skapa_kort(i, karaktärer(i))
    lägg_till(kortlista, kort1)
    lägg_till(kortlista, kort2)

return kortlista

program blanda_kort
blanda_kort_args(1) = kortlista

for i = längd(kortlista) to 2 step -1
    j = random(1, i)
    byt_plats(kortlista(i), kortlista(j))

return kortlista
\end{codebox}

Denna pseudokod följer de specificerade reglerna med naturligt språk, tydlig flödeshantering och korrekt variabelhantering. Koden är organiserad i distinkta program med tydliga in- och utargument.

\subsection*{Responsiv Design}
Spelets responsiva design implementeras genom CSS Grid och Flexbox:

\begin{codebox}
// Grundläggande layout
.game-board {
    display: grid;
    grid-template-columns: repeat(4, 1fr);
}

// Responsiv anpassning
@media (max-width: 600px) {
    .game-board {
        grid-template-columns: repeat(3, 1fr);
    }
}
\end{codebox}

\subsection*{Prestandaoptimering}
Koden inkluderar flera optimeringar för bättre prestanda:

\begin{codebox}
1. DOM-cachning:
   - Spara referenser till ofta använda element
   - Minimera DOM-sökningar

2. Event-delegering:
   - Använd en event-lyssnare för spelplanen
   - Delegera events till individuella kort

3. Animeringsoptimering:
   - Använd CSS transforms för animationer
   - Aktivera hardware-acceleration
   - Begränsa antal samtidiga animationer
\end{codebox}

Denna kodstruktur möjliggör enkel underhållning och utbyggnad av spelet, samtidigt som den säkerställer god prestanda och användarvänlighet.

\section*{AI-användning i Projektet}
I utvecklingen av detta projekt har artificiell intelligens använts som ett stödjande verktyg på följande sätt:

\subsection*{Bildgenerering}
\begin{itemize}
    \item DALL·E 3 användes för att generera unika fantasy-inspirerade bilder för spelkorten
    \item Detta säkerställde att bildmaterialet var fritt från upphovsrättsproblem
    \item Bilderna skapades med specifika instruktioner för att matcha spelets tema
\end{itemize}

\subsection*{Kodutveckling}
\begin{itemize}
    \item Claude (AI-assistent) användes som en pair-programming partner för att:
    \begin{itemize}
        \item Diskutera kodstruktur och arkitektoniska beslut
        \item Få förslag på optimeringar och förbättringar
        \item Identifiera potentiella problem och buggar
    \end{itemize}
    \item All kod granskades och validerades manuellt
    \item AI-förslagen anpassades för att matcha projektets specifika behov
\end{itemize}

\subsection*{Dokumentation}
\begin{itemize}
    \item AI assisterade i att:
    \begin{itemize}
        \item Strukturera dokumentationen
        \item Generera pseudokod för teknisk beskrivning
        \item Förbättra läsbarheten i tekniska förklaringar
    \end{itemize}
    \item All dokumentation har granskats och verifierats för korrekthet
\end{itemize}

Det är viktigt att notera att AI har använts som ett kompletterande verktyg, där alla beslut och implementationer har granskats och godkänts manuellt. AI:n har fungerat som en katalysator för idéer och en kvalitetssäkrare, medan den kreativa och tekniska visionen har styrts av mänskligt omdöme.

\end{document}
